\section*{Background} 
\addcontentsline{toc}{section}{Background}

Seismic assessment of existing structures is a complex procedure characterised by the definition of the expected seismic demand (hazard) and structural capacity (vulnerability). Generally, the \textit{seismic score} of the analysed structure is obtained by comparing demand and capacity. It is commonly accepted that seismic assessment is not a prediction of the actual performance of a structure under a particular seismic attack but instead is the general understanding of the likely behaviour of the structure under a certain level of the seismic demand.


\section*{Research motivations}
\addcontentsline{toc}{section}{Research motivations}

The SLaMA method originates from pioneering literature works regarding RC frames and cantilever wall structures , \cite{priestley1991}, \cite{park1995}, \cite{priestley1997}. The method was firstly introduced in the 2006 New Zealand seismic assessment guidelines. \cite{NZSEE2006}. In \cite{kam2013}, some practical considerations are drawn regarding to its application in seismic assessment of real buildings. Finally, a revamped version of SLaMA was introduced in the 2017 New Zealand guidelines on seismic assessment, \cite{NZSEE2016}, in which a tentative procedure for dual wall/frame systems was also introduced.


\section*{Hypothesis, objectives and scope}
\addcontentsline{toc}{section}{Hypothesis, objectives and scope}


This research work seeks to demonstrate the subsequent hypothesis: \\

\textit{SLaMA is a simple yet reliable method to obtain a first estimation of the capacity curve and the expected plastic mechanisms of an RC building with a Lateral Resisting System composed of frames (bare or masonry-infilled), cantilever walls and/or dual wall/frame systems.}\\


\section*{Research methodology and division of the work}
\addcontentsline{toc}{section}{Research methodology and division of the work}

To meet the objectives of this study, the work was divided into 4 phases, as shown in the Figure here below. 

Specific tasks were individuated for this work, one for each phase, and they are listed here below:



\section*{Dissertation outline}
\addcontentsline{toc}{section}{Dissertation outline}

Apart from this general \textbf{Introduction}, the dissertation is outlined as follows.\\

\textbf{Chapter \ref{chap:assessment}} gives a general overview and critical comparison of the code-based approaches for seismic vulnerability assessment of existing buildings, as suggested in the most important international standards. The conceptual approaches of three selected standards (EuroCode 8 - part 3, ASCE 41-13, NZSEE 2017) are compared. The mostly adopted analysis techniques are discussed, focusing on the different assumptions made in the selected standards/guidelines. \\

The \textbf{Conclusions} drawn from this PhD thesis work are finally listed, also recommending future developments.

